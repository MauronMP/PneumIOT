\documentclass[a4paper]{article}
\usepackage[utf8]{inputenc}
\usepackage[spanish]{babel}
\usepackage{graphicx}
\usepackage{subfigure}
\usepackage{url}
\usepackage{tabularx}
\begin{document}
    \begin{titlepage}
        \centering
        {\bfseries\LARGE Grado en Ingeniería Informática \par}
        \vspace{1cm}
        {\scshape\Large Escuela Técnica Superior de Ingenierías Informática y de Telecomunicaciones \par}
        \vspace{1cm}
        {\scshape\LARGE Componentes evaluables para el desarrollo del proyecto \par}
        \vspace{1cm}
        {\itshape\Large TFG \par}
        \vfill 
        {\Large Autor: \par}
        {\Large Pablo Morenilla Pinos \par}
        \vfill
        {\Large Marzo 2023 \par}
    \end{titlepage}

\clearpage

\section{Introducción.}
El objetivo de este documento es evaluar distintos aspectos con el fin de determinar que nos puede servir para el caso que queremos tratar, que es poder determinar que medidores, valores ambientales y condiciones poder evaluar de una habitación de hospital.
 
 En este documento y en el proyecto nos centramos concretamente en el área de neumología.
 Se irán viendo distintos campos a tener en cuenta, siendo el orden a seguir:
 \begin{enumerate}
 	\item Cómo es una habitación de hospital.
 	\item Elementos ambientales.
 	\item Sensores de arduino ambientales.
 	\item Determinar cuales pueden ser útiles.
 	\item Poder decidir cómo actuar dado los datos recibidos por los sensores.
 \end{enumerate}

\clearpage

\section{Cómo es una habitación de hospital.}

Las habitaciones de hospitales principalmente son de dos pacientes pero puede darse el caso de ser de uno o tres pacientes. Dependiendo del número de pacientes las medidas varían.

\begin{itemize}
	\item Habitaciones individuales(constan de una cama): Las medidas deben de ser sobre unos 10 metros cuadrados (esto varía dependiendo del país, en este caso nos centramos en España, concretamente Andalucía.)
	\item Habitaciones dobles(consta de dos camas): Las medidas deben de ser de unos 14 metros cuadrados.
	\item Habitaciones triples(constan de 3 camas): Las medidas pueden variar al igual que las anteriores, pero en este caso pueden ser además entre 18 a 20 metros cuadrados.
	
 \item Puede darse el caso ed que haya habitaciones de cuatro, pero este es el máximo permitido.
 
\end{itemize} 
Principalmente nos vamos a centrar en las habitaciones dobles e individuales.

En el caso de las camas, debe de existir una distancia lo suficientemente grande(entre 1 a 1,20 metros) entre las camas y la pared, para poder facilitar la atención al paciente.

La altura máxima de una habitación es de 2,5 metros.

Al espacio que hay en la habitación hay que restarle el espacio que ocupan los distintos elementos que hay en esta:

\begin{itemize}
	\item Mesillas.
	\item Mesa de cama.
	\item Silla o sillón.
	\item Papeleras
	\item Sofa de cama para acompañantes
	\item Tomas de oxígeno y vacío
	\item Baño
\end{itemize}

\clearpage

\section{Elementos ambientales en una habitación de hospital.}

A continuación se van a mostrar distintos elementos que condicionan el estado de una habitación de hospital.

\begin{itemize}
	\item Temperatura.
		\begin{itemize}
			\item La temperatura ambiental debe de estar entre unos 20 a 22 grados celsius, dependiendo de las zonas del hospital varía.
			\item Se regula por medio de los termostatos que disponen los pacientes.
			\item Dependiendo del hospital, puede disponer de un sistema de circuito cerrado de ventilación, el cuál lleva un sistema que, de manera automática, controla la temperatura.
		\end{itemize}
		
	\item Humedad.
		\begin{itemize}
			\item Se considera que está en un umbral adecuado cuando oscila entre el 40 y el 60 por ciento.
			\item En el caso de algunos estados patológicos pulmonares, el grado de humedad relativo bajo oscila entre el 10 y el 20 por ciento, para hacer que el paciente pueda estar en mejores condiciones.
			\item Se controla por medio de higrómetros que se encuentran en las unidades de los pacientes, también hay en pasillos y dependencias especiales.
		\end{itemize}
	\item Calidad del aire.
		\begin{itemize}
			\item La ventilación se realiza abriendo las ventanas y puerta durante breves periodos de tiempo. EL tiempo medio es de 10 a 15 minutos.
			\item La ventilación debe de hacerse de manera que no genere corrientes de aire para que así no sea de manera directa sobre el paciente.
			\item En los hospitales más modernizados, en caso de tener un circuito cerrado de aire acondicionado, se recomienda evitar abrir las ventanas para ventilar, ya que puede generar descompensaciones en el circuito del aire que está en constante renovación.
			\item Normalmente las impurezas que se encuentran en el aire son gases, particulas de polvo y microorganismos.
		\end{itemize}
	\item CO2.
		\begin{itemize}
			\item Los niveles de dióxido de carbono máximos recomendados en interiores debe de estar entre 400 y 800 ppm(partes por millón). 
			\item Para ello es recomendable renovar el aire de las habitaciones, siendo entonces la opción de abrir las ventanas o usando el circuito cerrado de aire en caso de disponer de este.
		\end{itemize}
	\item VOC (Volatile Organic Compounds).
		\begin{itemize}
			\item Compuestos orgánicos volátiles(COV) que hay en el aire en forma de gas o vapor a temperatura ambiente.
			\item Tiene distintas procedencias que pueden generar varios sintomas perjudiciales en los pacientes a través de la respiración y la piel.
				\begin{itemize}
					\item Náuseas.
					\item Dolor de cabeza.
					\item Mareos.
					\item Reacciones alérgicas.
				\end{itemize}
			\item Dependiendo del ppb(partículas por mil millones) puede ser:
				\begin{itemize}
					\item De 0 a 200. Muy bueno.
					\item De 201 a 600. Bueno.
					\item De 601 a 1000. Moderadamente malo.
					\item De 1001 a 2000. Muy malo.
					\item Más de 2000. Extremadamente perjudicial para la salud.		
				\end{itemize}
		\end{itemize}
	\item PM2.5.
		\begin{itemize}
			\item Se le denominan a aquellas partículas cuyo diámetro es igual o inferior a 2.5 micras. 
			\item Provienen de fuentes relacionadas con la actividad humana, como emisiones de gases contaminantes de vehículos, industris, agricultura entre otros. Puede provenir también por gases como SO2, NOx, NH3 y otros compuestos orgánicos volátiles.
			\item Tienen la particularidad de que son capaces de acceder a los pulmones e incluso alcanzar los alveolos, llevando sustancias perjudiciales a zonas sensibles del aparato respiratorio con riesgo a agravar enfermedades respiratorias.
			\item Los valores para el PM2,5 son:
			\begin{itemize}
				\item Bueno: menos de 25 µg/m3.
				\item Moderado: Entre 25 y 50 µg/m3.
				\item Malo: Entre 50 y 100 µg/m3.
				\item Muy malo: Entre 100 y 300 µg/m3.
				\item Extremadamente perjudicial: Más de 300 µg/m3.
			\end{itemize}
			\item Por otro lado, los valores legislados por el estándar son:
			\begin{itemize}
				\item 25 µg/m3 en 24 horas.
				\item 8 µg/m3 anuales.
			\end{itemize}
			\item La manera más adecuada de actuar en estos casos es como en las anteriores, ventilando el espacio del paciente, ya sea mediante el sistema de circuito de ventilación o abriendo las ventanas en caso de no tener el anterior.
		\end{itemize}
\end{itemize}
\end{document}
